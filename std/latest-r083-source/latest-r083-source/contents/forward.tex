\chapter*{说明}
\addcontentsline{toc}{chapter}{\hspace{-4em}说明}

\section*{数学记号}
函数变量普遍采用小写斜体字母,例如
\[q_1=f(p_1, \overline p_2, \cdots, \overline m)\]
其中$q_1$、$p_1$、分别表示作为内生变量的商品1的数量和价格,$\overline p_2$、$\overline m$分别表示作为外生变量的商品2的价格与消费者收入,$\cdots$ 表示其他未知的变量,$f(\cdot)$表示上述变量的函数关系。%当涉及较多的数学记号时往往会紧跟其后对其用法进行说明。

经济学量的书写遵循了常见教材的习惯,使用其英文单词首字母缩写进行简记并用大写斜体排版。所有经济学量在被定义或第一次被使用时都会对其英文名以及缩写进行说明,后文使用时一般使用缩写。例如\emph{边际成本}的英文名为 marginal cost 简记符号为$MC$,\emph{平均成本}的英文名为 average cost,简记符号为$AP$。

\section*{图表使用}
本笔记提供了大量的图示,虽然这种做法有填充版面之嫌,但图表在经济学原理分析中的重要性的的确确是不容忽视的。

在做比较静态分析的时候往往会涉及同一条曲线的移动,本笔记使用同种颜色不同浓度以示区别。其中浅色的表示初始状态的曲线,而同色较深的那支表示当前状态的曲线。例如第\pageref{fig:perunitquantitytax}页图(\ref{fig:perunitquantitytax})中浅蓝色曲线$D$表示征税之前的需求曲线,而深蓝色的$D_t$表示收到税收影响的需求曲线。

此外,我采用空心圆圈表示均衡分析中涉及的交点、切点、折点,这与数学习惯中去心邻域是无关的,如第\pageref{fig:profit-maxinization-condition}页图(\ref{fig:profit-maxinization-condition})位于横轴的空心圆圈$x_1$只表示等产量线$q^*$与横轴的交点。

\section*{参考文献}
注明出处总是必要的,这是一个好习惯,尊重原创者、为读者提供线索又为自己留足了推卸责任的余地。本书的参考文献均采用页内脚注的形式,毕竟这堆文字并非什么原创的知识体系,能在第一时间给想逃脱的人指明出口也算积德行善了。

\section*{在线资源}
我所能分享的也只有全书所有图形的源代码,主要的分享对象是高校年轻教师(不是青年教师)。年轻教师的定义是怀着理想的、未收深重污染的、从事教学却又顾家的年轻讲师。希望能为你们节省一点准备课件的时间。

所有的插图均采用\tikzname~\& \textsc{PGF}绘制。我的代码很不凝练,幸好有明确的注释,定性演示的话无需修改。


\section*{意见反馈}