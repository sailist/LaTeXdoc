\chapter{经济模型}
\label{sec:economic-models}
\pagenumbering{arabic}

经济数学模型一般是用由一组变量所构成的方程式或方程组来表示的,变量是经济模型的基本要素。变量可以被区分为内生变量、外生变量和参数。在经济模型中,内生变量指该模型所要决定的变量。外生变量指由模型以外的因素所决定的已知变量,它是模型据以建立的外部条件。

内生变量可以在模型体系内得到说明,外生变量决定内生变量,而外生变量本身不能在模型体系内得到说明。参数指数值通常不变的变量,也可以理解为可变的常数。参数通常是由模型以外的因素决定的,参数也往往被看成是外生变量。

经济模型可以被区分为静态模型和动态模型。从分析方法上讲,与静相联系的有静态分析方法和比较静态分析方法,与动态模型相联系的是动态分析方法。

仍以上面的均衡价格决定模型为例。在该模型中,当需求函数和供给的外生变量$\alpha$、$\beta$、$\delta$和$\gamma$被赋予确定数值以后,便可求出相应的均衡价$\bar P$均衡数量$\bar Q$的数值。这种根据既定的外生变量值来求得内生变量值的分析方法,静态分析。

在上述的均衡价格决定模型中,当外生变量$\alpha$、$\beta$、$\delta$和$\gamma$被确定为不同的数值时,由此得出的内生变量$\bar P$和$\bar Q$的数值是不相同的。很显然,在一个经济模型中,当外生变量的数值发生变化时,相应的内生变量的数值也会发生变化这种研究外生变量变化对内生变量的影响方式,以及分析比较不同数值的外生变量下的内生变量的不同数值,被称为比较静态分析。

\section{问题总结}
\subsection{为什么价格在纵轴,商品数量在横轴}
按照曼昆同志的说法我们都是马歇尔的学生,他老当初就是这么画的,我们也就因袭下来了。类似的表达在克鲁格曼和帕森斯那里也可以找到。\cite{gordon1982}

\subsection{经济学的研究对象}
以市场为资源配置基本方式的经济系统的构成要素
家庭或者消费者与厂商或者生产者之间的联系:产品市场和生产要素市场把家庭或者消费者与厂商或者生产者联系起来,共同构成一个有机的经济系统。
\subsection{微观经济学一个基本的假设条件}
以市场为资源配置基本方式的经济系统的构成要素
家庭或者消费者与厂商或者生产者之间的联系:产品市场和生产要素市场把家庭或者消费者与厂商或者生产者联系起来,共同构成一个有机的经济系统。
\begin{Definition}[理性人]\label{dfn:rationalman}\index{rational man 理性人}
“\emph{合乎理性的人}”简称“\emph{理性人}”或者“\emph{经济人}”,这种假设是对在经济社会中从事经济活动的所有人的基本特征的以一个一般性的抽象。其基本特征是:每一个从事经济活动的人都是利己的。或者说:每一个从事经济活动的人所采取的行为都是力图以自己最小的经济代价去获得最大的经济利益。经济学家认为在任何经济活动中,只有这样的人才是“合乎理性的人”。
\end{Definition}

\section*{推荐阅读}
\markright{推荐阅读}
\addcontentsline{toc}{section}{\hspace{-2.5em}推荐阅读}


\begin{asparaenum}
\item 杰克·赫舒拉发, 阿米亥·格雷泽, 大卫·赫舒拉发. {\itshape价格理论及其应用:决策、市场与信息}[M]. 李俊慧, 周燕, 译. 第7版. 北京: 机械工业出版社, 2009.% {\ding{224}} {\kaishu由于通常假定效用函数是连续的,所以,在同一坐标平面上的任何两条无差异曲线之间,可以有无数条无差异曲线。可以这样想像:我们可以画出无数条无差异曲线,以至覆盖整个平面坐标图。}
\item Varian, H. R. (1997). How to Build an Economic Model in Your Spare Time. {\itshape The American Economist}, 41(2), 3--10.% {\ding{224}} {\kaishu由于通常假定效用函数是连续的,所以,在同一坐标平面上的任何两条无差异曲线之间,可以有无数条无差异曲线。可以这样想像:我们可以画出无数条无差异曲线,以至覆盖整个平面坐标图。}
\end{asparaenum}