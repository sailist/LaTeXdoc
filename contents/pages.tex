\newpage
\section{页面处理}\label{sec:页面处理}
	\LaTeX 中的页面结构如下图所示:

    \layout\marginnote{该图使用\highunderline{layout}宏包,\highunderline{\textbackslash{}layout}命令生成}

    \LaTeX{}中各种命令的本质就是将文字、图片等“资源”放到一个个盒子中,并将这些盒子按一定的规则排布在页面上。文字默认都展示在Body主体中,如果需要设置页眉和页脚以及边注,就需要特殊的命令。

    % https://texblog.org/2011/04/04/make-page-margins-visible/ 显示页面边框。


    \subsection{页眉和页脚}
    设置页眉和页脚需要的宏包有\highunderline{fancyhdr}和\highunderline{titlesec}两种。经过多次测试后发现,\highunderline{fancyhdr}中的命令有时会会因为其他宏包的引入而失效,同时考虑到在使用中更改多种页面风格的需求,因此使用\highunderline{titlesec}在大多数情况下是一个更好的选择\footnote{要注意,该宏包和fancyhdr只能选择一个引入,否则会因为定义冲突而报错}。
    
    本节仅介绍titlesec\footnote{该宏包已开源,\href{https://github.com/jbezos/titlesec}{链接}}的使用方法。

    宏包引入:
    \begin{texcode}
        \usepackage{titlesec}
    \end{texcode}
    通过使用\highunderline{newpagestyle}来定义一种页面风格,定义分奇数页和偶数页两种,当版式不分奇偶时,则使用奇数页(奇数页是odd)的格式,定义方法如下:
    % \begin{texcode}
    %     \newpagestyle{stylename}{
    %         \sethead[evel-left][evel-center][evel-right]
    %                 {odd-left}{odd-center}{odd-right}
    %         \setfoot[evel-left][evel-center][evel-right]
    %                 {odd-left}{odd-center}{odd-right}
    %     }
    % \end{texcode}
    如:
    % \begin{texcode}
    %     \newpagestyle{mystyle}{%
    %         \sethead
    %         [第\thesection{}章~~\sectiontitle][][\thepage]%
    %         {第\thesection{}章~~\sectiontitle}{}{\thepage}
    %         \setfoot
    %         [][\LaTeX{}速查手册][]
    %         {}{\LaTeX{}速查手册}{}
    %     }
    % \end{texcode}
    
    使用定义好的页面风格\highunderline{需要在正文中声明},在序言区使用会出错。:
    % \begin{texcode}
    %     \begin{document}
    %         % \pagestyle{mystyle}
    %     \end{document}
    % \end{texcode}

    如果要在页眉或页脚处添加分界线,使用该宏包实现也非常的简单:
    \begin{texcode}
        \newpagestyle{mystyle}{%
            \headrule % 添加页眉处分割线
            \footrule % 添加页脚处分割线
            \setheadrule{1mm}%注意在声明后设置长度才有效(也可以不设置长度)
            \setfootrule{0.4pt}%0.4pt是分割线的默认值
            \sethead
            [第\thesection{}章~~\sectiontitle][][\thepage]%
            {第\thesection{}章~~\sectiontitle}{}{\thepage}
            \setfoot
            [][\LaTeX{}速查手册][]
            {}{\LaTeX{}速查手册}{}
        }        
    \end{texcode}

    \subsection{页版式与页边距}
    设置页版式和页边距等需要的包是\highunderline{geometry}
    % geometry宏包的版面
    \subsubsection{页面大小}
    geometry宏包提供默认的页面大小用于设置,如下:
    \begin{texcode}
        \usepackage[a4paper]{geometry}%用于控制版式,一般可以直接通过参数设置全局版式
        \usepackage[a4paper=true]{geometry}%用于控制版式,一般可以直接通过参数设置全局版式
        \usepackage[a4paper=false]{geometry}%用于控制版式,一般可以直接通过参数设置全局版式
    \end{texcode}
    注意赋值会被忽略掉,以上三种方式的效果是相同的。

    geometry提供的版式参数一共有6类,如下所示:
    \begin{center}
        % \newlength\tablewidth % if haven't define the length 'tablewidth'
        \setlength\tablewidth{\dimexpr (\textwidth -6\tabcolsep)}
        \begin{table}[H]
            \begin{tabular}{|p{0.33\tablewidth}<{\centering}|p{0.33\tablewidth}<{\centering}|p{0.33\tablewidth}<{\centering}|}
                \hline
                类型&示例&备注\\
                \hline
                a[0-6]paper&a0paper, a1paper...&ISOA标准\\
                \hline
                b[0-6]paper&b0paper, b1paper...&ISOB标准\\
                \hline
                c[0-6]paper&c0paper, c1paper...&ISOC标准\\
                \hline
                b[0-6]j&b0j, b1j, b2j...&JIS(Japanese Industrial Standards)B标准\\
                \hline
                ansi[a-e]paper&ansiapaper, ansibpaper...&未知\\
                \hline
                其他&letterpaper, executivepaper, legalpaper.&未知\\
                \hline
            \end{tabular}
        \end{table}
    \end{center}
    
    除了使用geometry提供的版面大小,还可以通过\highunderline{papersize}属性自定义纸张大小:
    \begin{texcode}
        \usepackage[papersize={20cm, 15cm}]{geometry}% {宽,高}
    \end{texcode}

    \subsection{纸张方向}
    在不改变版面大小的情况下,将所有页面的纸张横置只需要在添加\highunderline{geometry}包的时候添加\highunderline{landscape}参数:
    \begin{texcode}
        \usepackage[landscape]{geometry}    
    \end{texcode}

    如果需要单张的纸张更改纸张方向,实现方式比较多,但是都不能非常完美的完全更改过来(如页眉页脚在左右两侧等)。经过多个解决方案的尝试,该需求的最佳实践\footnote{参考自\href{https://tex.stackexchange.com/a/374533/193826}{StackExchange}}需要使用宏包\highunderline{etoolbox},并且在\highunderline{序言区}声明添加以下代码:
    \begin{texcode}
        % \usepackage{etoolbox}
        \makeatletter
        \def\ifGm@preamble#1{\@firstofone}
        \appto\restoregeometry{%
        \pdfpagewidth=\paperwidth
        \pdfpageheight=\paperheight}
        \apptocmd\newgeometry{%
        \pdfpagewidth=\paperwidth
        \pdfpageheight=\paperheight}{}{}
        \makeatother
    \end{texcode}
    随后,如果如果需要更改纸张方向,则在更改的地方使用以下命令(以默认竖向为例):
    \begin{texcode}
        \newgeometry{hmargin=3cm,vmargin=5cm,landscape} %前面两个间距的参数可以省略
    \end{texcode}
    随后的页面将完美的横置过来,直到想要重新恢复竖排时,使用以下命令:
    \begin{texcode}
        \restoregeometry
    \end{texcode}

        

    \subsection{文字竖排}
    由于\LaTeX{}主要用于学术用途,因此竖排的需求不是很常见。经查阅在\href{https://www.latexstudio.net/archives/536.html}{这篇文章}中有提到过如何实现文字竖排,留做坑,日后有需要再填。

    \subsection{换行}
    在\LaTeX{}中,单次换行并不会在排版中真正的换行:
    \begin{texshow}
        tex文件中的第一行
        tex文件中的第二行
    \end{texshow}

    如果需要换行,有四种方式,分别是空一行、使用双斜线、使用\highunderline{\textbackslash{}newline}命令和使用\highunderline{\textbackslash{}par}命令,这四种命令在一般情况下等价,如下:
    \begin{texshow}
        tex文件中的第一行

        tex文件中的第二行\\
        tex文件中的第三行\newline
        tex文件中的第四行\par
        tex文件中的第五行
    \end{texshow}
    
    另外,如果要\textbf{临时变化}与下一行的行间距,可以在\highunderline{\textbackslash{}\textbackslash{}}命令后添加\highunderline{[offset]}参数:
    \begin{texshow}
        tex文件中的第一行

        tex文件中的第二行\\[1em]
        tex文件中的第三行\newline %\newline 命令没有办法添加参数
        tex文件中的第四行
    \end{texshow}

    最后,还有一种换行方式\highunderline{\textbackslash{}linebreak},除了换行,还会将当前行分散对齐,效果如下:
    \begin{texshow}
        tex文件中的第一行\linebreak
        tex文件中的第二行\linebreak
        tex文件中的第三行
    \end{texshow}

    \subsection{换页}

    有时会遇到重新另起一页的需求(如目录结束后从新的一页开始),需要使用\highunderline{\textbackslash{}clearpage}或\highunderline{\textbackslash{}newpage}命令

    \begin{texcode}
        \tableofcontents
        \newpage
    \end{texcode}


    \subsection{分栏}
    分栏主要的需求是,全部分栏,部份分栏,在全部分栏中临时不分栏。

    如果需要全部分栏,只需要在初始文档类的声明中添加\highunderline{twocolumn}参数即可,如下:
    \begin{texcode}
        \documentclass[twocolumn]{article}
    \end{texcode}

    如果要对某一部分内容分栏,则需要使用\highunderline{multicols}环境\footnote{表格支持多行多列的宏包是multirow,两者只是名字相近}。

    \begin{texshow}
        \begin{multicols}{2}%列数
            多栏环境下,可以添加任何\LaTeX{}命令,但是不允许浮动,本手册中提及的长表格的使用页必须在单栏环境下才可以使用。
        \end{multicols}

        \begin{multicols}{3}
            多栏环境下,可以添加任何\LaTeX{}命令,但是不允许浮动,本手册中提及的长表格的使用页必须在单栏环境下才可以使用。
        \end{multicols}
    \end{texshow}

    可以通过添加参数的方式选择显示在分栏上方的文字:
    \begin{texshow}
        \begin{multicols}{2}[这里是分栏示例]
            多栏环境下,可以添加任何\LaTeX{}命令,但是不允许浮动,本手册中提及的长表格的使用页必须在单栏环境下才可以使用。
        \end{multicols}
    \end{texshow}

    列中间的分隔由\highunderline{\textbackslash{}columnsep}来控制,可以通过更改它的数值来控制间距:
    \begin{texshow}
        \setlength{\columnsep}{0.4pt}%在multicols环境前设置
        \begin{multicols}{2}[这里是分栏示例]
            多栏环境下,可以添加任何\LaTeX{}命令,但是不允许浮动,本手册中提及的长表格的使用页必须在单栏环境下才可以使用。
        \end{multicols}
    \end{texshow}

    同样,列中间也可以添加有颜色的分割线(默认黑色),来让分隔更清晰:
    \begin{texshow}
        \setlength{\columnseprule}{1pt}
        \renewcommand{\columnseprulecolor}{\color{red}}
        \begin{multicols}{3}[这里是分栏示例]
            多栏环境下,可以添加任何\LaTeX{}命令,但是不允许浮动,本手册中提及的长表格的使用页必须在单栏环境下才可以使用。
        \end{multicols}
    \end{texshow}

    在全多栏环境中,有时会需要插入跨栏的文字、表格或图片,对于表格和文字,\LaTeX{}提供了星号环境来实现多栏环境中的跨栏插入:
    \begin{texcode}
        \begin{figure*}
            ...
        \end{figure*}
        \begin{table*}
            ...
        \end{table*}
    \end{texcode}
    单栏文本在双栏环境中的插入要更困难一些,暂时没有太好的办法,只能使用\highunderline{\textbackslash{}onecolumn}临时声明,如下:
    \begin{texcode}
        \onecolumn 使用这种方法来插入部份的单栏文本
        \twocolumn 
    \end{texcode}
    注意必须在最后重新声明双栏环境,否则要么无法编译通过,要么之后会一直是单栏环境。
    
    \subsection{水印}
    % 有很多种实现方式 https://www.zhihu.com/question/35781113

    水印的实现方式有几种,各有优缺点,这里选择最方便容错性更高且支持中文的的宏包\highunderline{xwatermark}。

    该宏包使用只需要引入,然后定义水印即可,水印分文字水印和颜色水印两种,均可以定制大小、位置、显示页数等参数,同时可以多个水印同时使用,水印之间能够互相嵌套,非常方便。

    使用方法:
    \begin{texcode}
        \usepackage[printwatermark,disablegeometry]{xwatermark}

        \newwatermark[page=1,color=gray!20,angle=45,scale=4,xpos=0,ypos=0]
                     {SAILIST}
        \newwatermark[pages=2-4,color=gray!20,angle=45,scale=3,xpos=0,ypos=0]
                     {\LaTeX{}速查手册}
        \newwatermark[pagex={5,6,7},
                        color=gray!20,angle=45,scale=3,xpos=0,ypos=0]
                     {\LaTeX{}速查手册}
    \end{texcode}
    上述的代码已经在在本手册中执行,可以通过查看封面和目录页观察其效果。

    每种水印必须指定页数,该宏包支持的页数指定方式有8种:
    \begin{center}
        % \newlength\tablewidth % if haven't define the length 'tablewidth'
        \setlength\tablewidth{\dimexpr (\textwidth -8\tabcolsep)}
        \begin{table}[H]
            \begin{tabular}{|p{0.21\tablewidth}<{\centering}|p{0.26\tablewidth}<{\centering}|p{0.26\tablewidth}<{\centering}|p{0.26\tablewidth}<{\centering}|}
                \hline
                page=x&pages=x-y&pagex=\{x,y,z\}&firstpage\\
                \hline
                lastpage&allpages=true&evenpages=true&oddpages=true\\
                \hline
            \end{tabular}
            \caption{xwatermark宏包的页面指定方式}
        \end{table}
    \end{center}
    关于该宏包的更多使用方式,如图片水印等,参考该宏包的\href{http://ftp.jaist.ac.jp/pub/CTAN/macros/latex/contrib/xwatermark/doc/xwatermark-guide.pdf}{文档}
    
    % 横竖排放/