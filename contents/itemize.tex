\section{列表}\label{sec:列表}
    列表环境有三种,分别是\highunderline{itemize},\highunderline{enumerate},\highunderline{description}。
    \subsection{基本用法}
    生成没有序号的列表
    
    \begin{texshow}
        \begin{itemize}
            \item 第一个
            \item 第二个
            \item 第三个
            \begin{itemize}
                \item 第三个+1
                \item 第三个+2
            \end{itemize}
        \end{itemize}
    \end{texshow}
    
    生成有序号的列表
    \begin{texshow}
        \begin{enumerate}
            \item 第一个
            \item 第二个
            \item 第三个
            \begin{enumerate}
                \item 第三个+1
                \item 第三个+2
            \end{enumerate}
        \end{enumerate}
    \end{texshow}
    生成有标签描述的列表
    \begin{texshow}
        \begin{description}
            \item[itemize] 没有序号
            \item[enumerate] 有序号
            \begin{description}
                \item[罗马数字] 第一种
                \item[阿拉伯数字] ...
            \end{description}
            \item[description] 使用文字
        \end{description}
    \end{texshow}

    \subsection{自定义Bullet}
    "Bullet"是指各种List前的那个标记,中文译为"子弹",但觉得还是保留英文比较好一些。一般来说,默认的已经够用,但如果需要重新定义,\LaTeX{}同样支持高度可定制化的方式,另外,为了方便定制,可能需要使用\highunderline{enumitem}宏包\footnote{除了该宏包外,还有\highunderline{easylist},\highunderline{enumerate}等宏包可以完成,有需要可以自行了解}:
    \subsubsection{自定义标签}
    如果只是临时更改,那么只需要在\highunderline{\textbackslash{}item}后添加参数即可,如下:
    \begin{texshow}
        \begin{itemize}
            \item[--] 第一行
            \item[This] 第二行 %注意,虽然可以是文字,但缺少了description中的加粗,这是这两个环境的区别。
            \item 第三行 
        \end{itemize}
    \end{texshow}

    如果是统一更改一个itemize环境,那么可以在环境开始添加参数,如下:
    \begin{texshow}
        \begin{itemize}[label=--]
            \item 第一行
            \item 第二行
        \end{itemize}
    \end{texshow}

    如果是需要有一个经常使用的itemize环境,那么可以通过enumitem宏包直接定义一个,如定义一个\highunderline{todolist}(待完成列表):
    \begin{texshow}
        % \usepackage{enumitem}
        % \usepackage{amssymb} %\square 方法支持
        \newlist{todolist}{itemize}{2} % \newlist{<name>}{<type>}{<max-depth>}
        \setlist[todolist]{label=$\square$}
        
        \begin{todolist}
            \item 第一行
            \item 第二行
        \end{todolist}
    \end{texshow}

    上述的定义方法也可以合并为一个:
    \begin{texshow}
        \newlist{todolist2}{itemize}{2}
    \end{texshow}

    如果需要一个Bullet在多个环境里共用,那么可以新定义一个item,如下:
    \begin{texshow}
        \newcommand{\checkitem}{\item[$\square$]}%
        \begin{itemize}
            \item 第一行
            \checkitem 第二行
        \end{itemize}
    \end{texshow}

    我曾经因为需要定义了一个TOTOList,分享在这里:
    \begin{texshow}
        \begin{itemize}
            \item[$\square$]
            no checked
            \item[\rlap{\raisebox{0.3ex}{\hspace{0.4ex}\tiny \ding{52}}}$\square$]
            failed
            \item[\rlap{\raisebox{0.3ex}{\hspace{0.4ex}\scriptsize \ding{56}}}$\square$]
            checked
        \end{itemize}
    \end{texshow}
    \LaTeX{}中很难找到棱角分明的对号,只能用这种风格的折中一下。另外,关键在于Bullet如何用\LaTeX{}描述,如果需要很多使用,参考上面如何定义一个常用item。

    \subsubsection{自定义序号}
    序号的生成与\Ref{subsub:counter}和\Ref{subsub:number}有关,如果需要重新定义序号方式,方法与自定义标签类似,在单个item上重新定义需要借助计数器的名称,如:
    \begin{texshow}
        \begin{enumerate}
            \item 第一行 
            \stepcounter{enumi} % 如果指定了序号格式,需要使用之前手动对计数器加1
            \item[\alph{enumi}:] 第二行 
            \item 第三行 
            \begin{enumerate}
                \stepcounter{enumii}
                \item[\roman{enumii}] 二级+1
                \stepcounter{enumii}
                \item[\roman{enumii}] 二级+2 
            \end{enumerate}
        \end{enumerate}
    \end{texshow}

    在enumerate环境上的定义:
    \begin{texshow}
        \begin{enumerate}[label=\roman*)]
            \item 第一行
            \item 第二行
        \end{enumerate}
        \begin{enumerate}[label=\alph*--]
            \item 第一行
            \item 第二行
        \end{enumerate}
    \end{texshow}

    如果需要重新定义新的序号环境,方法同样与自定义标签类似:
    \begin{texshow}
        % \usepackage{enumitem}
        \newlist{romanlist}{enumerate}{2}
        \setlist[romanlist]{label=\roman*)}
        \begin{romanlist}
            \item 第一行
            \item 第二行
        \end{romanlist}
    \end{texshow}

    \subsection{版面与布局}
    \subsubsection{对齐与间隔}
    在使用\highunderline{enumitem}宏包后,就可以很方便的设置各种间距,有如下的几种:
    \begin{itemize}
        \item 垂直间距
        \begin{itemize}
            \item topsep
            \item partopsep
            \item parsep
            \item itemsep
        \end{itemize}
        \item 水平间距
        \begin{itemize}
            \item leftmargin
            \item rightmargin
            \item listparindent
            \item labelwidth
            \item labelsep
            \item itemindent
        \end{itemize}
    \end{itemize}
    
    可以通过附加参数的方式来设置,如:
    \begin{itemize}[itemsep=1ex, leftmargin=1cm]
        \item 示例文本示例文本示例文本示例文本示例文本示例文本
        \item 示例文本示例文本示例文本示例文本示例文本示例文本
    \end{itemize}
        
    如果要应用间隔设置到所有list,那么可以使用\highunderline{\textbackslash{}setlist}命令,如
    \begin{texshow}
        \newlist{romanlist}{enumerate}{2}
        \setlist[romanlist]{label=\roman*),leftmargin=1cm}
        \begin{romanlist}
            \item 第一行
            \item 第二行
        \end{romanlist}
    \end{texshow}
        
    \subsection{其他itemize布局}
    更多布局可以查看\href{https://en.wikibooks.org/wiki/LaTeX/List_Structures}{Wiki-List\_{}Structures}