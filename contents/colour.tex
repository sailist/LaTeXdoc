%! Author = saili
%! Date = 2019/8/25/0025

\section{颜色}\label{sec:color}
\subsection{引入}

如果想要更好的定义、使用和设置颜色,推荐使用%\highunderline[yellow]{xcolor}包
\begin{texcode}
\usepackage{xcolor}
\end{texcode}
xcolor包中定义了一些基本的颜色,有:

{\color{black}{black}},{\color{blue}{blue}}, {\color{brown}{brown}},{\color{cyan}{cyan}},{\color{darkgray}{darkgray}},{\color{gray}{gray}},{\color{green}{green}},{\color{lightgray}{lightgray}},{\color{lime}{lime}},{\color{magenta}{magenta}},{\color{olive}{olive}}

{\color{orange}{orange}},{\color{pink}{pink}},{\color{purple}{purple}},{\color{red}{red}},{\color{teal}{teal}},{\color{violet}{violet}},{\color{white}{white}},{\color{yellow}{yellow}}

另外xcolor中还定义了dvips中的68种标准色,如果要使用,需要在引入包时添加参数:
\begin{texcode}
\usepackage[dvipsnames]{xcolor}
\end{texcode}

\subsection{使用}
正如之前所使,几乎在任何一个可以更换颜色的地方,都可以更换相应位置的颜色。比如,{\color{red}{改变文字的颜色}}:
\begin{texshow}
{\color{red}这里放文字}\\
\textcolor{red}{这里放文字}%两者在设置文字颜色时等价
\end{texshow}

在大多数情况下,一些库或者自己生成的边框等,都可以通过设置其颜色属性或在外侧包裹color命令来实现变更颜色的方法。

\subsection{定义}
如果xcolor自带的颜色不足以使用,xcolor还提供了灵活的定义新颜色的方法,示例如下:

\begin{texcode}
\definecolor{light-gray}{gray}{0.95} %定义灰度
\definecolor{orange}{rgb}{1,0.5,0} %定义普通的rgb颜色(色值归一化为[0-1])
\definecolor{orange}{RGB}{255,127,0} %定义普通的rgb颜色(色值范围为0-255)
\definecolor{orange}{HTML}{FF7F00} %使用HTML设计时常用的方式,颜色由三个16进制数值组成
\definecolor{orange}{cmyk}{0,0.5,1,0} %使用印刷的色域,输入cmyk值
\end{texcode}

定义的颜色名字和xcolor自带的颜色名字具有相同的效果。

另外,在一些情况下,颜色还可以通过这样的一种语法来定义:
\begin{texcode}%默认底色是白色
    \colorlet{wblue}{blue!20} % 表示20%的蓝色+80%的白色
    \colorlet{bblue}{blue!20!black}% 表示20%的蓝色+80%的黑色
    \colorlet{bbg}{blue!20!black!30!green}% 表示20%的蓝色+30%的黑色+50%的绿色
\end{texcode}

在很多无法包裹color命令的情况下(如环境参数设置),大多是采用这种混合颜色(color mixes)的方式来设置颜色,如使用tcolorbox定义一个盒子:
\begin{texshow}
    \begin{tcolorbox}[colframe=blue!50!,colback=blue!20!]
        盒子内部
    \end{tcolorbox}
\end{texshow}

另外,如果要创建颜色的别名,也可以使用\highunderline{\textbackslash{}colorlet}:
\begin{texcode}
    \colorlet{mycolor}{someothercolor}
\end{texcode}

一般来说,新定义的颜色可以放到任何位置,但是为了规范,一般都倾向于将颜色统一定义在一个位置。

\subsection{其他环境下的颜色设置}

\subsubsection{表格}\label{sssec:color-table}
关于表格的颜色设置,参考\Ref{table-beauty}的说明

\subsection{超链接与引用}
超链接与引用的颜色,可以很方便的通过\highunderline[hlyellow]{hyperref}这个库来设置,只需要在导入库的时候定义参数即可,如:
\begin{texcode}
\usepackage[colorlinks,
linkcolor=black,
urlcolor=blue,
anchorcolor=blue,
citecolor=green]{hyperref}
\end{texcode}

\begin{texshow}
    \href{https://en.wikibooks.org/wiki/LaTeX/Hyperlinks}{Wiki-Hyperlinks}
\end{texshow}
关于设置的颜色对应哪个区域,以及更多的参数,可以参考\href{https://en.wikibooks.org/wiki/LaTeX/Hyperlinks}{Wiki-Hyperlinks}

\subsection{分割线}

\begin{texshow}
    %\rule[水平高度]{长度}{粗细}
    \color{red}{\rule[-10pt]{5cm}{0.05em}}
\end{texshow}

\subsection{颜色相关工具}
\subsubsection{LatexColor}
网站链接:\href{http://latexcolor.com/}{LatexColor}

点击看好的颜色直接复制定义颜色的语句,比较方便。
\begin{figure}[H]
    \centering
    \includegraphics[width=0.8\columnwidth]{\figpath{latexcolor.png}}
    \caption{latexcolor网站截图}
    \label{fig:latexcolor}
\end{figure}
