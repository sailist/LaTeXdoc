%! Author = saili
%! Date = 2019/8/26/0026

\section{单位与距离}\label{sec:大小与间距}

\subsection{单位}
\LaTeX{}中使用的单位可以分为长度和数字两种,长度单位如厘米、毫米等,数字单位是指数字的显示方式,将数字以原本形式展示(阿拉伯数字),以字母方式(a、b、c...),以罗马数字...等
\subsubsection{长度}
\LaTeX{}中的长度单位比较多,有以下几种:
\begin{center}
    % \newlength\tablewidth % if haven't define the length 'tablewidth'
    \setlength\tablewidth{\dimexpr (\textwidth -6\tabcolsep)}
    \begin{table}[H]
        \begin{tabular}{|p{0.19\tablewidth}<{\centering}|p{0.19\tablewidth}<{\centering}|p{0.62\tablewidth}<{\centering}|}
            \hline
            eng&含义&单位转换\\
            \hline
            mm&毫米&1 mm = 2.845 pt\\
            \hline
            pt&点&1 pt = 0.351 mm\\
            \hline
            bp&大点&1 bp = 0.353 mm >1 pt\\
            \hline
            dd&迪多&1 dd = 0.376 mm = 1.07 pt\\
            \hline
            pc&排卡&1 pc = 4.218 mm = 12 pt\\
            \hline
            sp&定标点&65536 sp = 1 pt\\
            \hline
            cm&厘米&1 cm= 10 mm= 28.453 pt\\
            \hline
            cc&西塞罗&1 cc= 4.513 mm= 12 dd = 12.84 pt\\
            \hline
            in&英寸&1 in = 25.4 mm = 72.27 pt\\
            \hline
            ex&ex&1 ex = 当前字体尺寸中 x 的高度\\
            \hline
            em&em&1 em = 当前字体尺寸中 M 的宽度\\
            \hline
        \end{tabular}
        \caption{\LaTeX{}中的间距}
        \label{tag:latex-distance}
    \end{table}
\end{center}

对水平距离的设置常用 $em$ ,而对垂直距离的设置,如行距,常用 $ex$。

\subsubsection{数字}\label{subsub:number}
\LaTeX{}中,经常会遇到数字格式的问题,如\highunderline{itemize}环境,或者章节前的序号,它们的格式实际上都是通过\LaTeX{}里的计数器(从0开始)、具体数值和数字风格转换方式来实现的,其中,默认的数字风格的转换有如下几种:
\begin{center}
    % \newlength\tablewidth % if haven't define the length 'tablewidth'
    \setlength\tablewidth{\dimexpr (\textwidth -8\tabcolsep)}
    \begin{table}[H]
        \begin{tabular}{|p{0.2\tablewidth}<{\centering}|p{0.29\tablewidth}<{\centering}|p{0.15\tablewidth}<{\centering}|p{0.33\tablewidth}<{\centering}|}
            \hline
            表示&含义&示例&备注\\
            \hline
            \textbackslash{}arabic&阿拉伯数字&0,1,2...&\\
            \hline
            \textbackslash{}roman&小写罗马数字&\romannumeral1,\romannumeral2,\romannumeral3...&\\
            \hline
            \textbackslash{}Roman&大写罗马数字&\uppercase\expandafter{\romannumeral1},\uppercase\expandafter{\romannumeral2},\uppercase\expandafter{\romannumeral3}...&\\
            \hline
            \textbackslash{}alph&小写字母&a,b,c...&范围为0-25\\
            \hline
            \textbackslash{}Alph&大写字母&A,B,C...&范围为0-25\\
            \hline
        \end{tabular}
        \caption{\LaTeX{}内置的几种表格}
    \end{table}
\end{center}

注意,它们的使用无法通过简单的在方法中嵌套数字来使用(如直接使用\highunderline{\textbackslash{}alph{0}}),而是需要结合计数器\footnote{有关更多计数器的内容,查看\Ref{subsub:counter}}来显示,如:
\begin{texshow}
    当前的章节号为\arabic{section}
\end{texshow}

对于中文支持,我们经常看到章节号会出现"一、二、三",这在\highunderline{ctex}的文档中提到,这其实是由\highunderline{zhnumber}包来完成的,其中有关计数器的内容主要是由\highunderline{\textbackslash{}chinese\{counter\}}来完成的,如:
\begin{texshow}
    当前的子章节号为\chinese{subsection}
\end{texshow}


\subsection{距离}
\subsubsection{距离变量的定义和设置}
\LaTeX{}排版中,有很多需要指定距离的需求,如果每次都使用长度单位,一旦更改会非常的麻烦,因此\LaTeX{}可以定义一个距离变量,方式如下:

\newlength\examplelen
\begin{texcode}
    \newlength\examplelen
\end{texcode}
注意:距离变量是唯一的,因此如果定义重名会报错。

距离变量定义好后,可以随时根据需要设置距离变量的长度,如下:
\begin{texshow}
    % 事先已经使用了 \newlength\examplelen
    \setlength{\examplelen}{3em}
    长度演示\hspace{\examplelen}长度演示\\
    长度演示\hspace{3em}长度演示\\

    \setlength{\examplelen}{2em}
    长度演示\hspace{\examplelen}长度演示
\end{texshow}

\subsubsection{常用距离}
\LaTeX{}中常用的距离有以下几种:
\begin{texcode}
    \linewidth - 当前行的宽度
    \columnwidth - 当前分栏的宽度
    \textwidth - 整个页面版芯的宽度
    \paperwidth - 整个页面纸张的宽度
\end{texcode}

在单栏文本中,\highunderline{\textbackslash{}columnwidth} 和 \highunderline{\textbackslash{}textwidth} 保持一致;在多栏文本中:

$$textwidth = n * columnwidth + (n - 1) * columnsep \text{\hspace{1em}(其中n是分栏数)}$$

在 \highunderline{minipage} 环境中,除了 \highunderline{\textbackslash{}paperwidth} 之外,其它三种都会根据 minipage\footnote{minipage的介绍和用法,参考\Ref{sub:minipage}} 的宽度发生改变(因为虚拟出了一个小的纸张页面),并且在 minipage 环境结束的时候恢复原样。

在\highunderline{parbox}中,\highunderline{\textbackslash{}textwidth} 和 \highunderline{\textbackslash{}columnwidth} 不会改变,不过 \highunderline{\textbackslash{}linewidth} 会发生变化。

\textbackslash{linewidth} 是相对最灵活的宽度值。在 list 环境里(包括 enumerate 和 itemize 等环境),在 \textbackslash{parbox} 里,\textbackslash{linewidth} 都会发生变化。
总的来说,当
\begin{itemize}
    \item 需要在列表环境中使用表格、图片等宽度的时候,用\highunderline{\textbackslash{}linewidth}
    \item 需要充满整个页面宽度的时候,用\highunderline{\textbackslash{}textwidth}(比如 figure/table 等)
    \item 需要充满整个分栏的时候,用\highunderline{\textbackslash{}columnwidth}(比如 figure/table/tabularx/tabu等)
\end{itemize}

% \todo{参考自https://liam.page/2015/08/17/width-in-latex/}

% \todo{查看注释}
% latex geometry 页面设置 https://www.jianshu.com/p/0719795278eb

\subsubsection{距离的运算}
所有的长度单位都可以通过在前面添加数字(正负、小数整数均可)来表示其倍数长度,如:
\begin{texcode}
    -2.5\textwidth        
\end{texcode}
如果是单位相同的两个长度,可以直接进行加减操作:
\begin{texcode}
    \textwidth-4\fboxsep
\end{texcode}
如果是不同长度的单位,可以使用\highunderline{dimexpr}宏包进行运算:
\begin{texcode}
    % \usepackage{dimexpr}
    \dimexpr (\textwidth -8\tabcolsep)
\end{texcode}
由于使用较少,因此关于该宏包的具体使用方法暂略。

\subsection{常用版面距离}\label{sub:normal-page-dist}
\subsubsection{字间距}
    % \todo{查看注释}
    一般没有调整字间距的需求,如果实在需要,可以查阅使用\highunderline{microtype}(用于英文)和\highunderline{ctex}宏包(用于中文)中的\highunderline{CJKglue}命令。\marginnote{这里有一篇\href{https://blog.csdn.net/lishoubox/article/details/49708095}{中文博客},但没有经过验证}
\subsubsection{行间距}
\begin{texcode}
    \linespread{1.3}%设置行距为1.3倍行距
\end{texcode}

\subsubsection{段间距}
\begin{texcode}
    \setlength{\parskip}{0.5em}
\end{texcode}

\subsubsection{首行缩进}
默认有首行缩进,如果需要默认首行不缩进,则使用\highunderline{indentfirst}包。

% \todo{tcolorbox下的indent效果被忽略的解决}
指定某段首行缩进(默认情况下不需要加,在使用\highunderline{indentfirst}宏包后如果要指定缩进可以使用):
\begin{texcode}
    % \usepackage{indentfirst}
    \indent 正常文本
\end{texcode}

指定某段首行不缩进,在段首加:
\begin{texcode}
    \noindent 正常文本
\end{texcode}

设置缩进量
\begin{texcode}
    \setlength\parindent{2em}
\end{texcode}

\subsubsection{悬挂缩进}
% \todo{悬挂缩进代码效果演示}

如果需要设置悬挂缩进,那么首先要取消首行缩进。
\begin{texcode}
    \noindent\hangafter=1 %设置从第一行之后开始悬挂缩进
\end{texcode}

设置悬挂缩进量:
\begin{texcode}
    \setlength{\hangindent}{2em}
\end{texcode}

\subsection{行之间插入空隙}
插入指定长度大小的空隙,可以使用\highunderline{\textbackslash{}vspace}:
\begin{texshow}
    %\vspace{宽度大小}和\vspace*{宽度大小}
    上一行\vspace{1cm}\\
    下一行\\
    正常间距测试
\end{texshow}

\subsection{设置目录间距}
\begin{texcode}
    \renewcommand{\baselinestretch}{0.75}\normalsize 
    \tableofcontents
    \renewcommand{\baselinestretch}{1.3}\normalsize %在设置后设置回原来的间距大小
\end{texcode}
